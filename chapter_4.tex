\chapter{%
結論}
本章では,まとめと今後の課題について述べる.

\section{まとめ}
本節では,本論文のまとめを述べる.
本研究では,既製品のロボットアームLite6に本研究室で開発された切り紙グリッパーの改良版を取り付け,物理モデルとデジタルモデルのロボットアームの手先位置および切り紙グリッパーの指先距離のデジタルツインを構築することを目的として,
ROSを用いてデジタルツインシステムを開発した.開発したシステムは物理モデルとデジタルモデルの制御を同時に行うことはできないが,Gazeboシミュレータ内に切り紙グリッパーを取り付けたLite6のデジタルモデルを構築することができた.
本論文では,はじめにROSデジタルツインシステムにおける物理モデルとデジタルモデルが何であるかを明記したあと,ROSデジタルツインシステムの構成とデジタルツインの内容を示した.システムの検証では定量的な検証方法を提案し,
その方法を用いて物理モデルとデジタルモデルのロボットアームの手先位置および切り紙グリッパーの指先距離を比較しROSデジタルツインシステムを評価した.
その結果,ロボットアームの手先位置と切り紙グリッパーの指先距離の制御は,実行時間においてはほとんど差はなく,位置制御においてもロボットアームの手先位置では数センチメートル,切り紙グリッパーの指先距離では数ミリメートルの差があったものの
,その値は小さくROSでデジタルツインシステムを開発できた.また,グリッパーの重心位置がロボットアームの手先制御に影響を与える可能性があることがわかった.

\section{今後の課題}
本節では,今後の課題を述べる.
現状,物理モデルとデジタルモデルの制御を同時に行うことはできないが,これは物理モデルとデジタルモデルを起動ファイル上でそれぞれのグループに分け2台のロボットをROSで制御するようにシステムを改良することでリアルタイム性のあるROSデジタルツインシステムを構築できる可能性があり,今後の課題である.